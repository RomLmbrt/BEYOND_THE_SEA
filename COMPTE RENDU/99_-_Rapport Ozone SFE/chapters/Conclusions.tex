%%%%%%%%%%%%%%%%%%%%%%%%%%%%%%%%%%%% Chapter Template

\chapter{Conclusions} 	% Main chapter title
\label{Conclusion} 		% For referencing the chapter elsewhere, usage \ref{Introduction}

% %%%%%%%%%%%%%%%%%%%%%%%%%%%%%%%%%%%% SECTION 1

% \section{Ozone Kitesurf}
% \label{sec:In1.1}

The primary objective of this internship was to develop Ozone's next-generation R1 formula kite for the 2028 Olympic Games. This endeavor began by utilizing the previous R1 kite version, R1 V4/V5, and the FlySurfer brand's VMG, known for its exceptional performance, as reference points. The overarching goal was to create the most high-performance kite possible for formula kite racing. 
Additionally, this internship aimed to introduce Ozone to a scientific approach to kite design, incorporating aerodynamic coefficients. Understanding the principles governing kite flight and translating these insights into aerodynamic characteristics proved beneficial in the pursuit of a more high-performance kite.

The process commenced with a comparative analysis of existing airfoils. This analysis allowed the experienced design team to discern the specific differences that contributed to the superior performance of the VMG kite compared to the R1 V4/V5. Subsequently, I employed Fluent simulations to meticulously design a new iteration of the R1 kite, resulting in the R1 V5 Satori 3 prototype. This prototype underwent extensive testing by our team riders and demonstrated superior performance compared to the VMG kite.

Furthermore, an optimization algorithm was developed using the Python library Aerosandbox. This algorithm was designed based on the same principles as XFLR5, serving as a preliminary airfoil design tool. It began with the R1 V5 Satori 3 as an initial reference and iteratively converged towards an even higher-performing kite design. Additionally, a 3D optimization algorithm was developed, building upon the principles of the 2D version but incorporating 3D effects and optimizing 3D parameters. While the 3D optimization program has not yet been utilized by Ozone, both algorithms offer valuable tools for the brand's future kite design endeavors.

The 2D optimization program has been employed to design several kites, not limited to the R1 formula kite, which was the focus of this internship. This demonstrates its flexibility in optimizing the parameters as desired by the designer. Furthermore, the 3D optimization program is being utilized by the paraglider design team on their considerably more powerful computers than mine. Should both teams, the paraglider design team and the kitesurf design team, collaborate and utilize this algorithm on their high-performance computers, it has the potential to yield significant benefits for the development of future Ozone kite versions.

In summary, the outcomes of this internship have played a pivotal role in assisting the Ozone team in crafting their next-generation formula kite, the R1 V5, slated for use in the 2028 Olympics. Furthermore, the implementation of optimization algorithms and the calculation of aerodynamic coefficients provide Ozone with a solid foundation for their future kite designs. 

Ultimately, this internship exposed me to the real-world challenges that engineers encounter within a company. The constraints of limited budgets also result in restricted simulation options, and addressing this challenge has encouraged me to explore creative and ingenious solutions. Additionally, the human aspects of water sports, including the sensations experienced by athletes when riding a formula kite, introduced an engineering dimension that I had not previously encountered but proved to be highly beneficial in expanding my open-mindedness.
