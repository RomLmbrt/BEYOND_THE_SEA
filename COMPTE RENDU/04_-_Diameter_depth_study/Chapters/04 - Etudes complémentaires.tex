%%%%%%%%%%%%%%%%%%%%%%%%%%%%%%%%%%%% Chapter 

\chapter{Etudes Complémentaires} 	
\label{Chapter4} 		

%%%%%%%%%%%%%%%%%%%%%%%%%%%%%%%%%%%%

%%%%%%%%%%%%%%%%%%%%%%%%%%%%%%%%%%%%%%%%%%%%%%%%%%%%%%%%%%%%%%%%%%%%%%%%%%%%%%%%
%%%%%%%%%%%%%%%%%%%%%%%%%%%%%%%%%%%% SECTION 1 %%%%%%%%%%%%%%%%%%%%%%%%%%%%%%%%%
%%%%%%%%%%%%%%%%%%%%%%%%%%%%%%%%%%%%%%%%%%%%%%%%%%%%%%%%%%%%%%%%%%%%%%%%%%%%%%%%

\section{Etude de la position de la cambrure maximale}
\label{sec:Ch4.1}

Tout d'abord Breukels ne prend en compte que le diamètre et la cambrure maximal d'un profil, \textbf{la position de la cambrure maximale d'un profil n'est donc pas prise en compte}. \\
En conduisant des simulations (VSM) basées cette fois sur des polaires déterminées à partir de NeuralFoil (Aersoandbox), on pense alors capturer l'influence de la position de la cambrure maximale. \textbf{On mène l'étude  sur des SK50-VG avec (t,k)=(0.06, 0.08)} :

\begin{figure}[H]
    \centering
    \includegraphics[width=\textwidth]{Pics/04 -Etudes complémentaires/position cambrure.png}  
    \caption{Influence de la position de la cambrure maximale}
    \label{fig:influence cambrure position}
\end{figure}

On observe que les résultats sont sensiblement les mêmes à des angles inférieurs à 10°. Cependant au delà de 10° NeuralFoil, qui est basé sur la même théorie que XFOIL (écoulement potentiel + équations couche limite) couplé avec de l'IA, perd en fiabilité. \\
\textbf{les résultats semblent donc être quasi-indentiques pour des variations de position de cambrure maximale entre 19\% et 25\%, aux faibles angles (<10°)}
%%%%%%%%%%%%%%%%%%%%%%%%%%%%%%%%%%%%%%%%%%%%%%%%%%%%%%%%%%%%%%%%%%%%%%%%%%%%%%%%
%%%%%%%%%%%%%%%%%%%%%%%%%%%%%%%%%%%% SECTION 2 %%%%%%%%%%%%%%%%%%%%%%%%%%%%%%%%%
%%%%%%%%%%%%%%%%%%%%%%%%%%%%%%%%%%%%%%%%%%%%%%%%%%%%%%%%%%%%%%%%%%%%%%%%%%%%%%%%

\section{Stabilité}
\label{sec:Ch4.2}

Ensuite, l'étude réalisée \textbf{ne fait pas intervenir de critère de stabilité.} En témoigne la figure \ref{fig:influence cambrure position}, nous avons ajouté l'évolution de Xcp, la position du centre de pousée le long de la corde, sur les polaires.\\
\textbf{\underline{Le raisonnement est le suivant} : "Xcp doit être suffisamment faible (effort aérodynamique proche du bord d'attaque) à faible Cl (limite de l'équilibre poids-portance) pour que le moment associé soit suffisamment cabreur et que l'aile ne tende pas à aller vers des Cl plus petits"}\\ 

\begin{figure}[H]
    \centering
    \begin{subfigure}[b]{0.45\textwidth}
        \centering
        \includegraphics[width=\linewidth]{Pics/04 -Etudes complémentaires/stabilité diam.png}
        \caption{$X_{CP}$ pour différent $\delta diamètre$}
        \label{fig:stabilité diamètre}
    \end{subfigure}
    \hfill
    \begin{subfigure}[b]{0.45\textwidth}
        \centering
        \includegraphics[width=\linewidth]{Pics/04 -Etudes complémentaires/stabilité dep.png}
        \caption{$X_{CP}$ pour différent $\delta cambrure$}
        \label{fig:stabilité cambrure}
    \end{subfigure}
    \caption{Sensibilité de la stabilité du kite aux variations de (t,k)}
    \label{fig:stabilité}
\end{figure}

Il semblerait donc que la cambrure joue un rôle majeur dans la stabilité du kite. Diminuer la cambrure déplace la courbe de $X_{CP}$ vers la gauche et donc stabilise le kite (figure \ref{fig:stabilité cambrure}). La figure \ref{fig:stabilité diamètre} montre la faible influence du diamètre de BA sur la stabilité. \\
\textbf{ Augumenter la cambrure semble donc dégrader la stabilité du kite. Ce résultat semble cohérent puisque augumenter la cambrure tend à augmenter la charge arrière. L'influence du diamètre est négligeable devant celle de la cambrure. }


%%%%%%%%%%%%%%%%%%%%%%%%%%%%%%%%%%%%%%%%%%%%%%%%%%%%%%%%%%%%%%%%%%%%%%%%%%%%%%%%
%%%%%%%%%%%%%%%%%%%%%%%%%%%%%%%%%%%% SECTION 3 %%%%%%%%%%%%%%%%%%%%%%%%%%%%%%%%%
%%%%%%%%%%%%%%%%%%%%%%%%%%%%%%%%%%%%%%%%%%%%%%%%%%%%%%%%%%%%%%%%%%%%%%%%%%%%%%%%

\section{Optimisation avec Aerosandbox}
\label{sec:Ch4.3}

Finalement les fonctions d'optimisation d'Aerosandbox ne sont pas applicables à la VSM. Elles sont cependant utilisées pour optimiser la fonction de Breukels (2 paramètres). \\
Aerosandbox peut aussi représenter et optimiser les profils avec 10 paramètres (Kulfan parameters). \textbf{Ce n'est pas le sujet de ce papier, cependant c'est une façon différente d'aborder le problème. Ce sera probablement le sujet d'une étude à venir.}

\begin{figure}[H]
    \centering
    \includegraphics[width=\textwidth]{Pics/04 -Etudes complémentaires/optim neuralfoil.png}  
    \caption{Optimisation réalisée avec Aerosandbox sur des profils dis "de Kulfan"}
    \label{fig:kulfan}
\end{figure}

%%%%%%%%%%%%%%%%%%%%%%%%%%%%%%%%%%%%%%%%%%%%%%%%%%%%%%%%%%%%%%%%%%%%%%%%%%%%%%%%
%%%%%%%%%%%%%%%%%%%%%%%%%%%%%%%%%%%% SECTION 4 %%%%%%%%%%%%%%%%%%%%%%%%%%%%%%%%%
%%%%%%%%%%%%%%%%%%%%%%%%%%%%%%%%%%%%%%%%%%%%%%%%%%%%%%%%%%%%%%%%%%%%%%%%%%%%%%%%

\section{Optimisation d'un SK5}
\label{sec:Ch4.4}

La même étude que présentée dans ce papier a été réalisée sur un SK5-VB afin de permettre un premier prototype pour Beyond. Voici les résulats :\\

\begin{figure}[H]
    \centering
    \begin{subfigure}[b]{0.45\textwidth}
        \centering
        \includegraphics[width=\linewidth]{Pics/04 -Etudes complémentaires/sk5 vb guassian diameter.png}
        \caption{$\delta$ diamètre selon les paramètres de la Gaussienne}
        \label{fig:diametre gaussien sk5 vb}
    \end{subfigure}
    \hfill
    \begin{subfigure}[b]{0.45\textwidth}
        \centering
        \includegraphics[width=\linewidth]{Pics/04 -Etudes complémentaires/sk5 vb guassian depth.png}
        \caption{$\delta$ cambrure selon les paramètres de la Gaussienne}
        \label{fig:cambrure gaussien sk5 vb}
    \end{subfigure}
    \caption{sk5 vb VB - Sensibilité des résultats 3D en fonction de l'écart-type $\sigma$ et de la valeur centrale $\alpha_{center}$ de la Gaussienne}
    \label{fig:gaussian sensibility sk5 vb}
\end{figure}

la figure \ref{fig:gaussian sensibility sk5 vb} montre qu'un choix de $\sigma$ = 8° et $\alpha_{center}$ = 10° permet de trouver un bon compromis d'optimum sur l'ensemble de la poalaire à faibles incidences. \\
Ainsi, pour ces paramètre, on obtiens la polaire suivante :\\

\begin{figure}[H]
    \centering
    \includegraphics[width=\textwidth]{Pics/04 -Etudes complémentaires/sk5 vb polar.png}
    \caption{Optimisation réalisée avec Aerosandbox sur des profils dis "de Kulfan"}
    \label{fig:polar sk5 vb}
\end{figure}

La figure \ref{fig:polar sk5 vb} montre que l'optimisation améliore en effet la polaire du kite à des incidences inférieures à 10° ainsi que sa stabilité.\\
\textbf{En conclusion :} $\alpha_{center}$ = 10°, $\sigma$ = 8°, ($\delta t_{opti}$, $\delta k_{opti}$) = (-0.0067, -0.0333) \\

Cela correspond à une répartition en envergure :
\begin{itemize}
    \item t : 0.063 0.068 0.073 0.073 0.073 0.073 0.073 0.073 0.073 0.073 0.073 0.073 0.073 0.073 0.073 0.073 0.073 0.073 0.068 0.063
    \item k : 0.032 0.034 0.037 0.037 0.037 0.037 0.037 0.037 0.037 0.037 0.037 0.037 0.037 0.037 0.037 0.037 0.037 0.037 0.034 0.032
\end{itemize}

A comparer avec le kite initial:
\begin{itemize}
    \item t : 0.085 0.091 0.090 0.0893 0.087 0.085 0.083 0.083 0.083 0.0820.083 0.085 0.087 0.089 0.090 0.091 0.085
    \item k : 0.042 0.050 0.065 0.069 0.072 0.075 0.08 0.08 0.08 0.08 0.08 0.075 0.072 0.069 0.065 0.050 0.042   
\end{itemize}

\textbf{Remarque :} SK5 VB V2 est identique à SK5 VB avec un diamètre de BA augumenté de 1.5\%. En menant une étude sur SK5 VB V2 on obtiens $\alpha_{center}$ = 10°, $\sigma$ = 8°, ($\delta t_{opti}$, $\delta k_{opti}$) = (+0.0067, -0.0333); ce qui est donc cohérent !\\

Si on trace l'évolution de ($\delta t_{opti}$, $\delta k_{opti}$) en fonction de $\alpha$ sans moyenner par une Gaussienne :\\

\begin{figure}[H]
    \centering
    \includegraphics[width=\textwidth]{Pics/04 -Etudes complémentaires/diam deth alpha sk5 vb.png}
    \caption{Evolution de ($\delta t_{opti}$, $\delta k_{opti}$) en fonction de alpha}
    \label{fig:diam deth alpha sk5 vb}
\end{figure}
