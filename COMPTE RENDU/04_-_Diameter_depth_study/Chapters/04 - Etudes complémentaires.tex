%%%%%%%%%%%%%%%%%%%%%%%%%%%%%%%%%%%% Chapter 

\chapter{Etudes Complémentaires} 	
\label{Chapter4} 		

%%%%%%%%%%%%%%%%%%%%%%%%%%%%%%%%%%%%

%%%%%%%%%%%%%%%%%%%%%%%%%%%%%%%%%%%%%%%%%%%%%%%%%%%%%%%%%%%%%%%%%%%%%%%%%%%%%%%%
%%%%%%%%%%%%%%%%%%%%%%%%%%%%%%%%%%%% SECTION 1 %%%%%%%%%%%%%%%%%%%%%%%%%%%%%%%%%
%%%%%%%%%%%%%%%%%%%%%%%%%%%%%%%%%%%%%%%%%%%%%%%%%%%%%%%%%%%%%%%%%%%%%%%%%%%%%%%%

\section{Etude de la position de la cambrure maximale}
\label{sec:Ch4.1}

Tout d'abord Breukels ne prend en compte que le diamètre et la cambrure maximal d'un profil, \textbf{la position de la cambrure maximale d'un profil n'est donc pas prise en compte}. \\
En conduisant des simulations (VSM) basées cette fois sur des polaires déterminées à partir de NeuralFoil (Aersoandbox), on pense alors capturer l'influence de la position de la cambrure maximale. \textbf{On mène l'étude  sur des SK50-VG avec (t,k)=(0.08, 0.05)} :

\begin{figure}[H]
    \centering
    \includegraphics[width=\textwidth]{Pics/04 -Etudes complémentaires/position cambrure.png}  
    \caption{Influence de la position de la cambrure maximale}
    \label{fig:influence cambrure position}
\end{figure}

On observe que les résultats sont sensiblement les mêmes à des angles inférieurs à 10°. Cependant au delà de 10° NeuralFoil, qui est basé sur la même théorie que XFOIL (écoulement potentiel + équations couche limite) couplé avec de l'IA, perd en fiabilité. \\
\textbf{les résultats semblent donc être quasi-indentiques pour des variations de position de cambrure maximale entre 19\% et 25\%, aux faibles angles (<10°)}
%%%%%%%%%%%%%%%%%%%%%%%%%%%%%%%%%%%%%%%%%%%%%%%%%%%%%%%%%%%%%%%%%%%%%%%%%%%%%%%%
%%%%%%%%%%%%%%%%%%%%%%%%%%%%%%%%%%%% SECTION 2 %%%%%%%%%%%%%%%%%%%%%%%%%%%%%%%%%
%%%%%%%%%%%%%%%%%%%%%%%%%%%%%%%%%%%%%%%%%%%%%%%%%%%%%%%%%%%%%%%%%%%%%%%%%%%%%%%%

\section{Stabilité}
\label{sec:Ch4.2}

Ensuite, l'étude réalisée \textbf{ne fait pas intervenir de critère de stabilité.} En témoigne la figure \ref{fig:influence cambrure position}, nous avons ajouté l'évolution de Xcp, la position du centre de pousée le long de la corde, sur les polaires.\\
\textbf{\underline{Le raisonnement est le suivant} : "Xcp doit être suffisamment faible (effort aérodynamique proche du bord d'attaque) à faible Cl (limite de l'équilibre poids-portance) pour que le moment associé soit suffisamment cabreur et que l'aile ne tende pas à aller vers des Cl plus petits"}


%%%%%%%%%%%%%%%%%%%%%%%%%%%%%%%%%%%%%%%%%%%%%%%%%%%%%%%%%%%%%%%%%%%%%%%%%%%%%%%%
%%%%%%%%%%%%%%%%%%%%%%%%%%%%%%%%%%%% SECTION 3 %%%%%%%%%%%%%%%%%%%%%%%%%%%%%%%%%
%%%%%%%%%%%%%%%%%%%%%%%%%%%%%%%%%%%%%%%%%%%%%%%%%%%%%%%%%%%%%%%%%%%%%%%%%%%%%%%%

\section{Optimisation avec Aerosandbox}
\label{sec:Ch4.3}

Finalement les fonctions d'optimisation d'Aerosandbox ne sont pas applicables à la VSM. Elles sont cependant utilisées pour optimiser la fonction de Breukels (2 paramètres). \\
Aerosandbox peut aussi représenter et optimiser les profils avec 10 paramètres (Kulfan parameters). \textbf{Ce n'est pas le sujet de ce papier, cependant c'est une façon différente d'aborder le problème. Ce sera probablement le sujet d'une étude à venir.}

\begin{figure}[H]
    \centering
    \includegraphics[width=\textwidth]{Pics/04 -Etudes complémentaires/optim neuralfoil.png}  
    \caption{Optimisation réalisée avec Aerosandbox sur des profils dis "de Kulfan"}
    \label{fig:kulfan}
\end{figure}

%%%%%%%%%%%%%%%%%%%%%%%%%%%%%%%%%%%%%%%%%%%%%%%%%%%%%%%%%%%%%%%%%%%%%%%%%%%%%%%%
%%%%%%%%%%%%%%%%%%%%%%%%%%%%%%%%%%%% SECTION 4 %%%%%%%%%%%%%%%%%%%%%%%%%%%%%%%%%
%%%%%%%%%%%%%%%%%%%%%%%%%%%%%%%%%%%%%%%%%%%%%%%%%%%%%%%%%%%%%%%%%%%%%%%%%%%%%%%%

\section{Optimisation d'un SK5}
\label{sec:Ch4.4}

La même étude que présentée dans ce papier a été réalisée sur un SK5-VB afin de permettre un premier prototype pour Beyond. Voici les résulats :\\