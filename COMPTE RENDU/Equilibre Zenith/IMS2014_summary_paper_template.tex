%% paper_template.tex is a modification of:
%% bare_conf.tex 
%% V1.2
%% 2002/11/18
%% by Michael Shell
%% mshell@ece.gatech.edu
%% 
%% This is a skeleton file demonstrating the use of IEEEtran.cls 
%% (requires IEEEtran.cls version 1.6b or later) with an IEEE conference paper.
%% 
%% Support sites:
%% http://www.ieee.org
%% and/or
%% http://www.ctan.org/tex-archive/macros/latex/contrib/supported/IEEEtran/ 
%%
%% This code is offered as-is - no warranty - user assumes all risk.
%% Free to use, distribute and modify.

% *** Authors should verify (and, if needed, correct) their LaTeX system  ***
% *** with the testflow diagnostic prior to trusting their LaTeX platform ***
% *** with production work. IEEE's font choices can trigger bugs that do  ***
% *** not appear when using other class files.                            ***
% Testflow can be obtained at:
% http://www.ctan.org/tex-archive/macros/latex/contrib/supported/IEEEtran/testflow


% Note that the a4paper option is mainly intended so that authors in
% countries using A4 can easily print to A4 and see how their papers will
% look in print. Authors are encouraged to use U.S. letter paper when 
% submitting to IEEE. Use the testflow package mentioned above to verify
% correct handling of both paper sizes by the author's LaTeX system.
%
% Also note that the "draftcls" or "draftclsnofoot", not "draft", option
% should be used if it is desired that the figures are to be displayed in
% draft mode.
%
% This paper can be formatted using the % (instead of conference) mode.
%++++++++++++++++++++++++++++++++++++++++++++++++++++++
%\documentclass[conference]{IEEEims} % Modified for MTT-IMS
%\documentclass[conference]{IMSTemplate}
\documentclass[conference]{IEEEtran}
%++++++++++++++++++++++++++++++++++++++++++++++++++++++
% If the IEEEtran.cls has not been installed into the LaTeX system files, 
% manually specify the path to it:
% \documentclass[conference]{../sty/IEEEtran} 


% some very useful LaTeX packages include:

%\usepackage{cite}      % Written by Donald Arseneau
                        % V1.6 and later of IEEEtran pre-defines the format
                        % of the cite.sty package \cite{} output to follow
                        % that of IEEE. Loading the cite package will
                        % result in citation numbers being automatically
                        % sorted and properly "ranged". i.e.,
                        % [1], [9], [2], [7], [5], [6]
                        % (without using cite.sty)
                        % will become:
                        % [1], [2], [5]--[7], [9] (using cite.sty)
                        % cite.sty's \cite will automatically add leading
                        % space, if needed. Use cite.sty's noadjust option
                        % (cite.sty V3.8 and later) if you want to turn this
                        % off. cite.sty is already installed on most LaTeX
                        % systems. The latest version can be obtained at:
                        % http://www.ctan.org/tex-archive/macros/latex/contrib/supported/cite/

%\usepackage{graphicx}  % Written by David Carlisle and Sebastian Rahtz
                        % Required if you want graphics, photos, etc.
                        % graphicx.sty is already installed on most LaTeX
                        % systems. The latest version and documentation can
                        % be obtained at:
                        % http://www.ctan.org/tex-archive/macros/latex/required/graphics/
                        % Another good source of documentation is "Using
                        % Imported Graphics in LaTeX2e" by Keith Reckdahl
                        % which can be found as esplatex.ps and epslatex.pdf
                        % at: http://www.ctan.org/tex-archive/info/
% NOTE: for dual use with latex and pdflatex, instead load graphicx like:
%\ifx\pdfoutput\undefined
%\usepackage{graphicx}
%\else
%\usepackage[pdftex]{graphicx}
%\fi
%+++++++++++++++++++++++++++++++++++++++++++
% Added to commands
\input epsf
\usepackage{graphicx}
\usepackage [french]{babel}
\usepackage [utf8]{inputenc}
\usepackage [T1] {fontenc}
\usepackage{float}
%+++++++++++++++++++++++++++++++++++++++++++
% However, be warned that pdflatex will require graphics to be in PDF
% (not EPS) format and will preclude the use of PostScript based LaTeX
% packages such as psfrag.sty and pstricks.sty. IEEE conferences typically
% allow PDF graphics (and hence pdfLaTeX). However, IEEE journals do not
% (yet) allow image formats other than EPS or TIFF. Therefore, authors of
% journal papers should use traditional LaTeX with EPS graphics.
%
% The path(s) to the graphics files can also be declared: e.g.,
% \graphicspath{{../eps/}{../ps/}}
% if the graphics files are not located in the same directory as the
% .tex file. This can be done in each branch of the conditional above
% (after graphicx is loaded) to handle the EPS and PDF cases separately.
% In this way, full path information will not have to be specified in
% each \includegraphics command.
%
% Note that, when switching from latex to pdflatex and vice-versa, the new
% compiler will have to be run twice to clear some warnings.


%\usepackage{psfrag}    % Written by Craig Barratt, Michael C. Grant,
                        % and David Carlisle
                        % This package allows you to substitute LaTeX
                        % commands for text in imported EPS graphic files.
                        % In this way, LaTeX symbols can be placed into
                        % graphics that have been generated by other
                        % applications. You must use latex->dvips->ps2pdf
                        % workflow (not direct pdf output from pdflatex) if
                        % you wish to use this capability because it works
                        % via some PostScript tricks. Alternatively, the
                        % graphics could be processed as separate files via
                        % psfrag and dvips, then converted to PDF for
                        % inclusion in the main file which uses pdflatex.
                        % Docs are in "The PSfrag System" by Michael C. Grant
                        % and David Carlisle. There is also some information 
                        % about using psfrag in "Using Imported Graphics in
                        % LaTeX2e" by Keith Reckdahl which documents the
                        % graphicx package (see above). The psfrag package
                        % and documentation can be obtained at:
                        % http://www.ctan.org/tex-archive/macros/latex/contrib/supported/psfrag/

%\usepackage{subfigure} % Written by Steven Douglas Cochran
                        % This package makes it easy to put subfigures
                        % in your figures. i.e., "figure 1a and 1b"
                        % Docs are in "Using Imported Graphics in LaTeX2e"
                        % by Keith Reckdahl which also documents the graphicx
                        % package (see above). subfigure.sty is already
                        % installed on most LaTeX systems. The latest version
                        % and documentation can be obtained at:
                        % http://www.ctan.org/tex-archive/macros/latex/contrib/supported/subfigure/

%\usepackage{url}       % Written by Donald Arseneau
                        % Provides better support for handling and breaking
                        % URLs. url.sty is already installed on most LaTeX
                        % systems. The latest version can be obtained at:
                        % http://www.ctan.org/tex-archive/macros/latex/contrib/other/misc/
                        % Read the url.sty source comments for usage information.

%\usepackage{stfloats}  % Written by Sigitas Tolusis
                        % Gives LaTeX2e the ability to do double column
                        % floats at the bottom of the page as well as the top.
                        % (e.g., "\begin{figure*}[!b]" is not normally
                        % possible in LaTeX2e). This is an invasive package
                        % which rewrites many portions of the LaTeX2e output
                        % routines. It may not work with other packages that
                        % modify the LaTeX2e output routine and/or with other
                        % versions of LaTeX. The latest version and
                        % documentation can be obtained at:
                        % http://www.ctan.org/tex-archive/macros/latex/contrib/supported/sttools/
                        % Documentation is contained in the stfloats.sty
                        % comments as well as in the presfull.pdf file.
                        % Do not use the stfloats baselinefloat ability as
                        % IEEE does not allow \baselineskip to stretch.
                        % Authors submitting work to the IEEE should note
                        % that IEEE rarely uses double column equations and
                        % that authors should try to avoid such use.
                        % Do not be tempted to use the cuted.sty or
                        % midfloat.sty package (by the same author) as IEEE
                        % does not format its papers in such ways.

%\usepackage{amsmath}   % From the American Mathematical Society
                        % A popular package that provides many helpful commands
                        % for dealing with mathematics. Note that the AMSmath
                        % package sets \interdisplaylinepenalty to 10000 thus
                        % preventing page breaks from occurring within multiline
                        % equations. Use:
%\interdisplaylinepenalty=2500
                        % after loading amsmath to restore such page breaks
                        % as IEEEtran.cls normally does. amsmath.sty is already
                        % installed on most LaTeX systems. The latest version
                        % and documentation can be obtained at:
                        % http://www.ctan.org/tex-archive/macros/latex/required/amslatex/math/



% Other popular packages for formatting tables and equations include:

%\usepackage{array}
% Frank Mittelbach's and David Carlisle's array.sty which improves the
% LaTeX2e array and tabular environments to provide better appearances and
% additional user controls. array.sty is already installed on most systems.
% The latest version and documentation can be obtained at:
% http://www.ctan.org/tex-archive/macros/latex/required/tools/

% Mark Wooding's extremely powerful MDW tools, especially mdwmath.sty and
% mdwtab.sty which are used to format equations and tables, respectively.
% The MDWtools set is already installed on most LaTeX systems. The lastest
% version and documentation is available at:
% http://www.ctan.org/tex-archive/macros/latex/contrib/supported/mdwtools/


% V1.6 of IEEEtran contains the IEEEeqnarray family of commands that can
% be used to generate multiline equations as well as matrices, tables, etc.


% Also of notable interest:

% Scott Pakin's eqparbox package for creating (automatically sized) equal
% width boxes. Available:
% http://www.ctan.org/tex-archive/macros/latex/contrib/supported/eqparbox/



% Notes on hyperref:
% IEEEtran.cls attempts to be compliant with the hyperref package, written
% by Heiko Oberdiek and Sebastian Rahtz, which provides hyperlinks within
% a document as well as an index for PDF files (produced via pdflatex).
% However, it is a tad difficult to properly interface LaTeX classes and
% packages with this (necessarily) complex and invasive package. It is
% recommended that hyperref not be used for work that is to be submitted
% to the IEEE. Users who wish to use hyperref *must* ensure that their
% hyperref version is 6.72u or later *and* IEEEtran.cls is version 1.6b 
% or later. The latest version of hyperref can be obtained at:
%
% http://www.ctan.org/tex-archive/macros/latex/contrib/supported/hyperref/
%
% Also, be aware that cite.sty (as of version 3.9, 11/2001) and hyperref.sty
% (as of version 6.72t, 2002/07/25) do not work optimally together.
% To mediate the differences between these two packages, IEEEtran.cls, as
% of v1.6b, predefines a command that fools hyperref into thinking that
% the natbib package is being used - causing it not to modify the existing
% citation commands, and allowing cite.sty to operate as normal. However,
% as a result, citation numbers will not be hyperlinked. Another side effect
% of this approach is that the natbib.sty package will not properly load
% under IEEEtran.cls. However, current versions of natbib are not capable
% of compressing and sorting citation numbers in IEEE's style - so this
% should not be an issue. If, for some strange reason, the user wants to
% load natbib.sty under IEEEtran.cls, the following code must be placed
% before natbib.sty can be loaded:
%
% \makeatletter
% \let\NAT@parse\undefined
% \makeatother
%
% Hyperref should be loaded differently depending on whether pdflatex
% or traditional latex is being used:
%
%\ifx\pdfoutput\undefined
%\usepackage[hypertex]{hyperref}
%\else
%\usepackage[pdftex,hypertexnames=false]{hyperref}
%\fi
%
% Pdflatex produces superior hyperref results and is the recommended
% compiler for such use.



% *** Do not adjust lengths that control margins, column widths, etc. ***
% *** Do not use packages that alter fonts (such as pslatex).         ***
% There should be no need to do such things with IEEEtran.cls V1.6 and later.


% correct bad hyphenation here
\hyphenation{op-tical net-works semi-conduc-tor IEEEtran}
\begin{document}

% paper title
%\title{Submission Format for IPVC-CyberSec21 (Title in 24-point Times font)}
% If the \LARGE is deleted, the title font defaults to  24-point.
% Actually, 
% the \LARGE sets the title at 17 pt, which is close enough to 18-point.
%+++++++++++++++++++++++++++++++++++++++++++
\title{\LARGE Equilibre au Zénith du Kite
\vskip10pt

\small 2024 - Romain LAMBERT
}
%+++++++++++++++++++++++++++++++++++++++++++
% use only for invited papers
%\specialpapernotice{(Invited Paper)}

% make the title area
\maketitle

\begin{abstract}
Ce bureau d'étude a pour sujet la résolution numérique de la stabilité d'un système oscillant couplé en élévation et tangage. Une expérience n'a pas pu permettre de confirmer tous les résultats théoriques et il convient de calibrer notre code sur la seule partie réussie de l'expérience et de la théorie. Une étude de l'erreur inhérente au code et donc sa capacité de prédiction sera aussi menée. 
\end{abstract}
\IEEEoverridecommandlockouts
%%\begin{keywords}
%%Ceramics, coaxial resonators, delay filters, %%delay-lines, power
%%amplifiers.
%%\end{keywords}
%% no keywords
% For peer review papers, you can put extra information on the cover
% page as needed:
% \begin{center} \bfseries EDICS Category: 3-BBND \end{center}
% for peerreview papers, inserts a page break and creates the second title.
% Will be ignored for other modes.
\IEEEpeerreviewmaketitle
\section{Validation du modèle découplé}
% no \PARstart

\textbf{Question 1 -  Fréquences naturelles} 
\\ \\
En considérant séparément le PFD sur la plaque plane et l'équation des moments on peut trouver les fréquences naturelles théoriques de pilonnement et de tangage : 
\begin{center}
    $f_{th,pilonnement} = \frac{1}{2\pi}\sqrt{\frac{K_h}{m}} =  7.04 Hz$ 
\end{center}
\begin{center}
    $f_{th,tangage} = \frac{1}{2\pi}\sqrt{\frac{K_\theta}{I}} = 9.00 Hz$
\end{center}

Lorsqu'on se place dans le cas où on bloque un mode pour essayer de retrouver ces valeurs. On trouve respectivement pour les modes : 

\begin{center}
    $f_{pilonnement} = 7.2 Hz$
\end{center}
\begin{center}
    $f_{tangage} = 9.0 Hz$
\end{center}

On remarque que la fréquence en tangage est exactement (à la précision du résultat près) le même que celle théorique mais que celle en pilonnement a une erreur de $2,3\%$
\\ \\
Ces frénquences sont obtenues de la sorte :
\newline
1/ On calcule ta FFT du mouvement obtenu 
\newline
2/ On scinde de spectre en 2 pour éviter les effets de repliement du spectre en fréquence
\newline
3/ On regarde la fréquence du mode d'amplitude maximale
\\ \\
Il est difficile de comprendre pourquoi seul un mode est teinté d'erreur, celle-ci provient surement de la résolution numérique (les inversions de matrice induisant souvent des imprécisions selon les moyens que l'on s'est donné pour les calculer). 
\\ \\
Enfin, on remarque aussi qu'à chaque fois le programme ne nous donne pas une fréquence nulle pour le mode bloqué. Cela provient surement du bruit des signaux obtenus donnant des pics qui vont être ressortis par l'analyse de la FFT. 

\pagebreak

\textbf{Question 2 - Vitesse de divergence en statique } 

\begin{center}
    $q_{max} = \frac{1}{2}\rho_0 U_c^2 = \frac{K_\theta}{SC.C_{M_\theta}}$
\end{center}
 avec $S$ la surface de notre plaque, $C$ la corde, et $C_{M_\theta} = 2\pi (\frac{x_f}{C} - \frac{1}{4})$

\hfill

Donc $U_{c,div} = \sqrt{\frac{2.K_\theta}{\rho_0SC.2\pi(\frac{x_f}{C} - \frac{1}{4})}} = 23,91 m/s $

Avec le code on trouve une valeur de de divergence statique de $23,9 m/s$ ce qui est une erreur quasi nulle ($<1\%$) : le code est donc vérifié pour le cas statique. 

\begin{figure}[H]
  \includegraphics[width=\linewidth]{Pics/Vitesse divergence statique.PNG}
  \caption{Tracé de $\theta$ en fonction du temps pour $U_{inf} = 23.9m/s$, première vitesse pour laquelle $\theta$ diverge}
  \label{fig:boat1}
\end{figure}

\section{Système couplé avec modèle aéro simple}

\textbf{Question 3 - Vitesse de flutter }

Avec les approximations faites, on obtient le système suivant : 
\newline
\begin{center}
    (1) $m \ddot h + K_h h = qS2\pi \theta$ et (2)  $I \ddot \theta + K_\theta \theta = q\frac{SC}{4} 2\pi \theta $
\end{center}


Une fois mis sous forme matriciel on obtient deux valeur propre, que l'on souhaite égale (par définition de la vitesse de flutter). 

On aboutit à $U_c = \sqrt{(\frac{K_\theta}{I}-\frac{K_h}{m})\frac{4I}{SC\pi \rho_0}} = 14,75 m/s$

Et donc $U_c^* =  \frac{U_\infty}{\omega_\theta C} = 7,45$

On trouve cependant une valeur inférieur avec le modèle numérique, $U_c = 11,5m/s$. Cet écart peut s'expliquer parce qu'on calcule théoriquement la vitesse pour laquelle la divergence apparait alors que celle trouvée sur le code est la première où les fréquences sont égales.


\begin{figure}[H]
  \includegraphics[width=\linewidth]{Pics/Q3 U_flutter.png}
  \caption{Tracé des oscillations pour $U_{inf} = 11.5m/s$, première vitesse pour laquelle on a égalité des fréquences}
  \label{fig:boat1}
\end{figure}

Cependant on retrouve bien le fait que la vitesse $U_c = 14,75 m/s$ est la borne inférieur à partir de laquelle on a une divergence. 

\begin{figure}[H]
  \includegraphics[width=\linewidth]{Pics/Flutter_speed.png}
  \caption{Tracé des oscillations pour $U_{inf} = 14.75m/s$, première vitesse pour laquelle on a une divergence sur le code}
  \label{fig:boat1}
\end{figure}




\textbf{Question 4 - Vitesse de flutter avec modèle quasi-statique }

En passant en quasi-steady, on ne peut plus repérer la vitesse de $U_{flutter}$ en repérant la vitesse pour laquelle on a synchronisation des fréquences. En effet, en quasi-steady, on repère la vitesse de battement en regardant la vitesse conduisant à une divergence des grandeurs de notre système. Notre critère de divergence étant que soit $h$, soit $\theta$ dépasse leurs valeurs initiales avec une allure exponentielle.  Ce critère étant arbitraire, nous n'avons conservé qu'une précision de $0.5m/s $ sur notre vitesse obtenue. 

En modifiant le modèle de $\theta$ en prenant un compte sa dépendance en mouvement verticaux, on remarque que l'on la vitesse de flutter baisse et se trouve maintenant à $10m/s$.

\begin{figure}[H]
  \includegraphics[width=\linewidth]{Pics/Q4 U_flutter.png}
  \caption{Tracé des oscillations pour $U_{inf} = 10m/s$, seuil à partir duquel les amplitudes divergent}
  \label{fig:boat1}
\end{figure}

Ceci s'explique par le fait que la vitesse de pilonnement additionnelle vient créer une incidence supplémentaire sur le profil et ainsi générer plus de portance pour une même vitesse $U_c$. Le phénomène de flutter étant lié à la portance générée, il va donc arriver à une vitesse plus faible que dans notre ancienne configuration. 

De plus on note une modification de la fréquence de synchronisation qui augmente de 5\% 


\textbf{Question 5 - Approche Wagner}

Dans notre equation (1) on note la présence de $\theta$ et dans l'équation (2), en réécrivant $\theta = \alpha + \frac{\dot h }{V_\infty}$ on se rend compte du couplage entre les deux équations.

En rajoutant le terme $\Delta L$ induit par le retard prévu par la théorie de Wagner dans l'expression de la portance, on obtient la même valeur de flutter de $10m/s$.

\begin{figure}[H]
  \includegraphics[width=\linewidth]{Pics/Code Wagner.PNG}
  \caption{Lignes de code qui prennent en compte la théorie de Wagner}
  \label{fig:boat1}
\end{figure}

\begin{figure}[H]
  \includegraphics[width=\linewidth]{Pics/Wagner.png}
  \caption{Tracé des oscillations pour $U_{inf} = 10m/s$, nouvelle valeur de $U_{flutter}$}
  \label{fig:boat1}
\end{figure}

Ceci semble étrange au premier abord puisqu'on a apporté une correction; cependant on se rend compte que le retard apporté est très faible aux fréquences auxquelles notre système évolue. Ainsi aux échelles de temps auxquelles on travaille sur notre plaque, l'approche quasi-steady semble donner un bon ordre de grandeur. 

\section{Approche complète}

\textbf{Question 6 - Approche Theodorsen}

En lisant le papier concernant l'expérience abordée dans ce BE, on trouve les nouvelles expressions de la portance et du moment dans la théorie de Therodorsen : 


\begin{figure}[H]
    \includegraphics[width=\linewidth]{Pics/Theodorsen_Theory.png}
\end{figure}

On trouve une valeur de la vitesse de flutter de $9m/s$, plus grande que sur le modèle de Wagner donc. Ceci est normal car là où Wagner prend en compte le wake par une fonction de retard dans le domaine temporel, Theodorsen utilise une masse additionnlle qui va engendrer une inertie supplémentaire et donc un amortissement supplémentaire. Il faut donc une vitesse plus grande de $U_{inf}$ pour atteindre la divergence. 

\begin{figure}[H]
  \includegraphics[width=\linewidth]{Pics/Theodorsen_mass.png}
  \caption{Modèle théorique de Theodorsen}
  \label{fig:boat1}
\end{figure}


En revanche on trouve une fréquence de synchronisation égale ce qui est encourageant dans le sens où le modèle de Wagner certes moins poussé permet quand même de retrouver certains résultats importants. 

\begin{figure}[H]
  \includegraphics[width=\linewidth]{Pics/Theodorsen_Q6.png}
  \caption{Nouvelle vitesse de $U_{flutter}$ à $9.0m/s$ avec le modèle de Theodorsen}
  \label{fig:boat1}
\end{figure}

Quand on regarde dans le cours on retrouve la forme des termes circulatoires et non-circulatoires : 

\begin{figure}[H]
    \includegraphics[width=\linewidth]{Pics/Theodorsen_Cours.png}
\end{figure}

On peut alors les isoler et regarder leur proportions en fonction du déplacement vertical et de l'angle de plaque (on se place à chaque fois dans des cas où $U_{inf} = U_{flutter}$ pour avoir des comportements similaires) : 

\begin{figure}[H]
  \includegraphics[width=\linewidth]{Pics/Circ = f(Theta).png}
  \caption{Proportion de portance dû à la circulation (rouge) et évolution de $\theta$ (bleue) }
  \label{fig:boat1}
\end{figure}

\begin{figure}[H]
  \includegraphics[width=\linewidth]{Pics/Circ = f(h).png}
  \caption{ Proportion de portance dû à la circulation (rouge) et évolution de h (bleue)}
  \label{fig:boat1}
\end{figure}

On voit que la partie circulatoire du terme de portance varie avec les oscillations du système. La partie circulatoire passe la majorité du temps majoritaire mais s'annule lorsque $\theta$ est maximal et $h = 0$. En effet, lorsque $h = 0$,$\dot h$ est maximal. 

\textbf{Question 7 - Approche Theodorsen avec couplage et amortissement }

\begin{figure}[H]
  \includegraphics[width=\linewidth]{Pics/Q7 U_flutter S0.png}
  \caption{Nouvelle vitesse de $U_{flutter}$ à $8.5m/s$ avec le modèle de Theodorsen et $S_{0}$ non nul}
  \label{fig:boat1}
\end{figure}

\hfill

\begin{figure}[H]
  \includegraphics[width=\linewidth]{Pics/Circ = f(h) S0.png}
  \caption{Nouvelle vitesse de $U_{flutter}$ à $9.5m/s$ avec le modèle de Theodorsen et $S_{0}$, $K_{h}$ et $K_{\theta}$ non nuls}
  \label{fig:boat1}
\end{figure}

On voit qu'en incluant les amortissements, la vitesse amont conduisant au phénomène de battement augmente. Cela semble normal puisqu'on dissipe plus d'énergie et qu'il faut donc une vitesse amont plus grande. 


\textbf{Question 8 - Conclusion }

Au cours de ce BE nous avons abordés plusieurs méthodes de résolution du sytème plaque plane avec une approche couplée entre les efforts sur la structure et les efforts aérodynamiques, dans les cas stationnaires et instationnaires. 
\newline
Les approches de Wagner et de Theodorsen permettent par des méthodes différentes d'approcher les résultats du cas instationnaire. On a constaté avec la méthode de Wagner qu'aux fréquences de notre système la correction était faible et que le modèle quasi-steady est plutôt bon. Après réflexion une fréquence de $8Hz$ peut sembler faible mais cela représente quand même 8 oscillations de notre profil en 1secondes. Difficile de comprendre pourquoi cela a peu d'influence par rapport au modèle quasi-steady.
\newline
Enfin nous sommes allés plus loin sur la méthode de Theodorsen en implementant un couplage supplémentaire et en prenant en compte les termes d'amortissement et en obeservant qu'ils augmentent les vitesses amonts conduisant au battement du système. 

% \begin{table*}
% \centerline { TABLE 1  } 
% \vskip5pt
% \centerline { Summary of Typographical Settings}
% \vskip2pt
% \centerline{
% \vbox{\offinterlineskip
% \hrule
% %\vskip2pt\hrule\vskip2pt
% % Leading & means preamble template repeats infinitely. p.241 TeX Book.
% \halign{&\vrule#&
% \strut\quad#\hfil\quad\cr
% %Use either first and third lines following this description, OR the
% %second line.  The first choice is used when all vertical rules go to the
% %top of the first horizontal line of the table.  The second choice below
% %(with the \strut) is used when there are column headings that span
% %more than one column.  The \strut in that column line will not have the
% %vertical tic marks in the horizontal rule.  Note that a vrule is also
% %considered a column, so when using \multispanx, x is the number of
% %all columns including the ``vrule.'' 
% %height2pt&\omit&&\omit&&\omit&&\omit&&\omit&&\omit&&\omit&&\omit&&\omit&\cr
% &\strut &&\multispan5\hfil {\bf Font Specifics}\hfil&&\multispan9\hfil {\bf Paragraph Description}\hfil &\cr
% %&\omit &&\multispan5\hfil {\bf Font Specifics}\hfil&&\multispan9\hfil {\bf Paragraph Description}\hfil &\cr
% &{\bf Section}&&\multispan5\hfil (Times Roman unless
% specified)\hfil&&\multispan5\hfil spacing (in points)\hfil &&
% alignment&&indent&\cr
% &\omit&&style&&size&&special&&line&&before&&after&&\omit&&(in inches)&\cr
% height2pt&\omit&&\omit&&\omit&&\omit&&\omit&&\omit&&\omit&&\omit&&\omit&\cr
% \noalign{\hrule}
% height2pt&\omit&&\omit&&\omit&&\omit&&\omit&&\omit&&\omit&&\omit&&\omit&\cr
% %\noalign{\vskip2pt\hrule\vskip2pt}
% %\omit&\omit&\omit&\omit\cr
% &Title&&plain&&18&&none&&single&&6&&6&&centered&&none&\cr
% &Autohr List&&plain&&12&&mpme&&single&&6&&6&&centered&&none&\cr
% &Affiliations&&plain&&12&&none&&single&&6&&6&&centered&&none&\cr
% &Abstract&&bold&&9&&none&&exactly 10&&0&&0&&justified&&0.125 $1^{st}$ line&\cr
% &Headings&&plain&&10&&small caps&&exactly 12&&18&&6&&centered&&none&\cr
% &Subheadings&&italic&&10&&none&&exactly 12&&6&&6&&left&&none&\cr
% &Body&&plain&&10&&none&&exactly 12&&0&&0&&justified&&0.125 $1^{st}$ line&\cr
% &Paragrahps&&\omit&&\omit&&\omit&&\omit&&\omit&&\omit&&\omit&&\omit&\cr
% &Equations&&\multispan5 \hfil Symbol font for special characters
% \hfil&&single&&6&&6&&centered&&none&\cr
% &Figures&&\multispan5 \hfil 6 to 9 point sans serif (Helvetica)\hfil&&single&&0&&0&&centered&&none&\cr
% &Figure Captions&&plain&&9&&none &&10&&0&&0&&justified&&none, tab at 0.5&\cr
% &References&&plain&&9&&none&&10&&0&&0&&justified&&0.25 hanging&\cr
% height2pt&\omit&&\omit&&\omit&&\omit&&\omit&&\omit&&\omit&&\omit&&\omit&\cr}
% \hrule}}
% \end{table*}


% \begin{eqnarray}
% \oint {\bf E \cdot dL} & = & - {\partial\over \partial t}
% \int\!\!\!\int {\bf B \cdot} d {\bf S}\\
% \noalign{\hbox{or}}
% \nabla \times {\bf H} & = & {\bf J} + {\partial {\bf D} \over \partial t}.
% \end{eqnarray}

%Within Microsoft Word there are several options for placing figures
%within your paper. Often the easiest is to insert them between
%existing paragraphs allowing the figures to remain in that relative
%position. The paragraph description where the figure is inserted must
%be set to ``single'' spacing rather than ``exactly 12 points'' in
%order to allow the line to autoscale in height to display the entire
%figure. Some disadvantages of this approach are that you don't have
%total flexibility in placing figures, and that the figures will move
%as text is inserted or deleted in any part of the document before the
%figure. If you elect to use this approach, it is recommended that you
%nearly complete the editing of your text before inserting any
%figures. Remember to allow room for them, however. Then begin
%inserting figures starting from the beginning of your document. 

%More flexibility is obtained in inserting figures if you can place
%them exactly where you would like them to be on a page. This can be
%accomplished by inserting the figure, selecting the figure, and then
%choosing ``Format Picture\dots ''. Various settings allow you to place the figure at an absolute position on a page; specify if the text is supposed to flow around the figure or if the figure should move with the text, etc. If you elect to let the text flow around the figure, then remember that you will have to insert a separate text box for the caption, otherwise the figure caption is likely to become separated from the figure.
%When importing a graph from Excel into Word, it is often helpful to
%special-paste it in as a ``Picture (Enhanced Meta-file).'' This saves
%file memory for Word documents. Be aware that the usual Copy
%$\rightarrow $  Paste procedure will copy the entire Excel spreadsheet into your Word file. The Copy $\rightarrow$ Paste Special $\rightarrow$ Picture (Enhanced Metafile) command copies only the graph as a static picture. This is not a concern with PDF file submissions.

%\smallskip
%\begin{figure}
%\epsfxsize=3.25in\epsfbox{figure2.epsi}
%\caption{Example of an improperly titled figure. The numerics and the labels on the axes are illegible. This will cause a submission to be rejected. Don't let this happen to you!}
%\end{figure}
%\smallskip

\hfill 


% When referencing a journal article \cite{cantrell1}, a conference
% digest article \cite{cantrell2} or a book \cite{krauss}, place the reference numbers within square
% brackets. To simultaneously cite these references \cite{cantrell1} - \cite{krauss} use the format just demonstrated.  
% \cite{cantrell1} - \cite{krauss} as a guide. 

%Further information on LaTeX and TeX can be found in \cite{IEEEhowto:kopka} - \cite{knuth}. 

% The following statement makes the two columns on the last page more
% or less of equal length.  Placement of this command is by trial and error.
%\vfil\eject


%\subsection{Subsection Heading Here}
%\subsubsection{Subsubsection Heading Here}

% Reminder: the "draftcls" or "draftclsnofoot", not "draft", class option
% should be used if it is desired that the figures are to be displayed while
% in draft mode.

% An example of a floating figure using the graphicx package.
% Note that \label must occur AFTER (or within) \caption.
% For figures, \caption should occur after the \includegraphics.
%
%\begin{figure}
%\centering
%\includegraphics[width=2.5in]{myfigure}
% where an .eps filename suffix will be assumed under latex, 
% and a .pdf suffix will be assumed for pdflatex
%\caption{Simulation Results}
%\label{fig_sim}
%\end{figure}


% An example of a double column floating figure using two subfigures.
%(The subfigure.sty package must be loaded for this to work.)
% The subfigure \label commands are set within each subfigure command, the
% \label for the overall fgure must come after \caption.
% \hfil must be used as a separator to get equal spacing
%
%\begin{figure*}
%\centerline{\subfigure[Case I]{\includegraphics[width=2.5in]{subfigcase1}
% where an .eps filename suffix will be assumed under latex, 
% and a .pdf suffix will be assumed for pdflatex
%\label{fig_first_case}}
%\hfil
%\subfigure[Case II]{\includegraphics[width=2.5in]{subfigcase2}
% where an .eps filename suffix will be assumed under latex, 
% and a .pdf suffix will be assumed for pdflatex
%\label{fig_second_case}}}
%\caption{Simulation results}
%\label{fig_sim}
%\end{figure*}



% An example of a floating table. Note that, for IEEE style tables, the 
% \caption command should come BEFORE the table. Table text will default to
% \footnotesize as IEEE normally uses this smaller font for tables.
% The \label must come after \caption as always.
%
%\begin{table}
%% increase table row spacing, adjust to taste
%\renewcommand{\arraystretch}{1.3}
%\caption{An Example of a Table}
%\label{table_example}
%\begin{center}
%% Some packages, such as MDW tools, offer better commands for making tables
%% than the plain LaTeX2e tabular which is used here.
%\begin{tabular}{|c||c|}
%\hline
%One & Two\\
%\hline
%Three & Four\\
%\hline
%\end{tabular}
%\end{center}
%\end{table}
%\begin{table}
%\caption{An Example of a Table}
%\label{table_example}
%\begin{center}
%\begin{tabular}{|c||c|}
%\hline
%One & Two\\
%\hline
%Three & Four\\
%\hline
%\end{tabular}
%\end{center}
%\end{table}


% conference papers do not normally have an appendix

% use section* for acknowledgment

% optional entry into table of contents (if used)
%\addcontentsline{toc}{section}{Acknowledgment}


% trigger a \newpage just before the given reference
% number - used to balance the columns on the last page
% adjust value as needed - may need to be readjusted if
% the document is modified later
%\IEEEtriggeratref{8}
% The "triggered" command can be changed if desired:
%\IEEEtriggercmd{\enlargethispage{-5in}}

% references section
% NOTE: BibTeX documentation can be easily obtained at:
% http://www.ctan.org/tex-archive/biblio/bibtex/contrib/doc/

% can use a bibliography generated by BibTeX as a .bbl file
% standard IEEE bibliography style from:
% http://www.ctan.org/tex-archive/macros/latex/contrib/supported/IEEEtran/bibtex
%\bibliographystyle{IEEEtran.bst}
% argument is your BibTeX string definitions and bibliography database(s)
%\bibliography{IEEEabrv,../bib/paper}
%
% <OR> manually copy in the resultant .bbl file
% set second argument of \begin to the number of references
% (used to reserve space for the reference number labels box)


 % BIBLIOGRAPHIE
% \begin{thebibliography}{1}


% \bibitem {cantrell1}
% W. H. Cantrell, ``Tuning analysis for the high-Q class-E power
% amplifier,'' \emph{IEEE Trans. Microwave Theory \& Tech.}, vol. 48,
% no. 12, pp. 2397-2402, December 2000.

% \bibitem {cantrell2}
% W. H. Cantrell, and W. A. Davis, ``Amplitude modulator utilizing a
% high-Q class-E DC-DC converter'', \emph {2003 IEEE MTT-S Int. Microwave
% Symp. Dig.}, vol. 3, pp. 1721-1724, June 2003.

% \bibitem {krauss}
% H. L. Krauss, C. W. Bostian, and F. H. Raab, \emph{Solid State Radio Engineering}, New York: J. Wiley \& Sons, 1980.

% %\bibitem{IEEEhowto:kopka}
% %H.~Kopka and P.~W. Daly, \emph{A Guide to {\LaTeX}}, 3rd~ed.\hskip 1em plus
% % 0.5em minus 0.4em\relax Harlow, England: Addison-Wesley, 1999.

% %\bibitem{lamport} L. Lamport, \emph{ {\LaTeX} A Document Preparation
% %  System}, Reading, Mass: Addison-Wesley, 1994.

% %\bibitem{knuth} D. E. Knuth, \emph {The \TeX book}, Reading, Mass.:
% %  Addison-Wesley, 1996.

% \end{thebibliography}
% \smallskip
% Note: For the Summary paper submission only, references to the authors own work should be cited as if done by others to enable a double-blind review. {\bfseries Citations must be complete and not redacted, allowing the reviewers to confirm that prior art has been properly identified and acknowledged.}
% % that's all folks
\end{document}